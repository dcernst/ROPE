\documentclass[11pt]{article}
\pagestyle{empty}

\usepackage{graphicx}
\usepackage{color}

\oddsidemargin=0in
\textwidth=6.5in
\textheight=9in
\topmargin=-.625in
\linespread{1.25}

\begin{document}
\begin{center}
{\Large \textbf{Collaborative Research: Resource of Open\\
Problems for Education (ROPE)}}
\end{center}

\bigskip
\noindent
The proposed project will develop an on-line, electronic library, the Resource of Open Problems for Education (ROPE), that will
provide many innovative, well-tested, and documented problems
that instructors and students can use in a wide range of courses and assignment types.  Any user will be able to search this
library for problems by course, topic, type of problem (e.g.,
computational, conceptual, etc.), level of difficulty, and other
characteristics.  Problems will be contributed by the community of users
and will have associated descriptive information that will include user
feedback and comments, as well as statistics on the frequency of the
problem's use.  Users will be able to rate problems using a commonly
understood ``like'' feature (similar to those used on social networking
sites), create and share collections of problems they may refer to later
for homework, quizzes or exams, and comment on problems for the benefit of
other users.  The library will support different problem formats, the
ability to translate between them, and will be able to interact with other
applications (for example, on-line homework systems) to provide problems
for them.

\bigskip\bigskip
\noindent
{\large \textbf{Intellectual Merit}}

The Resource of Open Problems for Education (ROPE) will address a current need in
undergraduate mathematics education: the need for a widely-available
source of good ({\color{red}{what do we mean by ``good"?}}) problems that instructors can use in a variety of
educational venues.  Much of the learning that takes place in mathematics
is driven by students' work, and the success of that learning is
fundamentally dependent on the types of problems on which they work.  We expect ROPE to be a significant and widely-used tool for
mathematics educators.  Its search interface, open nature, user feedback,
and extensibility will result in an easy-to-use resource for faculty that will 
accordingly have a significant impact on student learning of undergraduate
mathematics. {\color{red} How will we measure this?}

\bigskip\bigskip
\noindent
{\large \textbf{Broader Impacts}}

The broad impact of ROPE will stem from its accessibility, ease-of-use
and extensibility.  We expect that the largest group of users of ROPE
will be faculty who are browsing for additional homework, test or quiz
problems for their courses.  ROPE will provide problems for users and
authors of open-content and conventionally published textbooks, and will
be particularly useful for open-content texts.  Because ROPE will be
designed to be flexible and extensible, we can
meet the changing needs of faculty in the future and make connections with
other software projects for which a library of open problems will be
useful.  We expect that ROPE will have broad impact, reaching
instructors at all types of colleges and universities who are teaching any
standard mathematics course in any of a number of different ways.  We
further expect that as ROPE develops we will be able to extend it to
include other disciplines, further increasing its impact.

\end{document}
