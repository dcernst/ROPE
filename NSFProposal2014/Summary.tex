\documentclass[11pt]{article}
\pagestyle{empty}

\renewcommand{\rmdefault}{phv} % Arial
\renewcommand{\sfdefault}{phv} % Arial

\usepackage{graphicx}
\usepackage{color}

\oddsidemargin=0in
\textwidth=6.5in
\textheight=9in
\topmargin=-.625in
\linespread{1.25}

\begin{document}
\begin{center}
{\Large \textbf{Collaborative Research: Resource of Open\\
Problems for Education (ROPE)}}
\end{center}

\bigskip
\noindent
The proposed project will develop an on-line, electronic library, the Resource of Open Problems for Education (ROPE), that will
provide many innovative, well-tested, and documented problems
that mathematics instructors and students can use in a wide range of courses and assignment types.  Any user will be able to search this
library for problems by course, topic, type of problem (e.g.,
numerical, symbolic, etc.), level of difficulty, and other
characteristics; and by keyword or problem text.  Problems will be contributed by the community of users
and will have associated descriptive information that will include user
feedback and comments, as well as statistics on the frequency of the
problem's use.  Users will be able to rate problems using a commonly
understood ``endorse'' feature (similar to those used on social networking
sites), create and share collections of problems they may refer to later
for homework, quizzes or exams, and comment on problems for the benefit of
other users.  The library will support different problem formats, the
ability to translate between them, and will be able to interact with other
applications (for example, on-line homework systems) to provide problems
for them.

\bigskip\bigskip
\noindent
{\large \textbf{Intellectual Merit}}

The Resource of Open Problems for Education (ROPE) will address a current need in
undergraduate mathematics education: the need for a widely-available
source of high-quality, well documented problems that instructors can use
in a variety of educational venues.  Much of the learning that takes place in mathematics
is driven by students' work, and the success of that learning is
fundamentally dependent on the types of problems on which they work.  ROPE
will provide a resource that will be \emph{free and open-source},
providing problems for users without commercial entanglements.  Because it
will have a powerful search capability, users will be able to easily find
problems that allow them to address their needs for instruction or
learning.  It will allow users to create \emph{collections of problems},
and \emph{collections of collections}.  This capability will allow many
individual use-cases: for example, instructors may construct homework
sets, quizzes, and groups of these for entire courses; or may construct
model courses with supporting material that they can then share with
colleagues; etc.  And it will support a \emph{community of users} who may
contribute problems, content and feeback on problems, and who may share
their work and problem collections with others.  All of these
characterstics, taken in sum, will result in a widely accessible and
useful resource that may have a significant impact on mathematics
education as a whole.

\bigskip\bigskip
\noindent
{\large \textbf{Broader Impacts}}

The broad impact of ROPE will stem from its accessibility, ease-of-use and
extensibility.  We expect that the largest group of users of ROPE will be
faculty who are browsing for additional homework, test or quiz problems
for their courses, and constructing collections for their courses.  ROPE
will also provide problems for users and authors of open-content and
conventionally published textbooks, and will be particularly useful for
open-content texts.  The PIs for the project have extensive contacts with
faculty in a large number of institutions and communities of faculty,
including Project NExT, those interested in Inquiry Based Learning, and
open-content authors, which will allow for effective dissemination to
those groups most likely to use the resource.  Because it will support
flexible collections of problems and collections, the number of ways
instructors will be able to use it will be broad, and it should have
commensurately broad impact.  In addition, the flexibility and
extensibility of its design will mean that ROPE can meet the changing
needs of faculty in the future and make connections with other software
projects for which a library of open problems will be useful.  We
therefore expect that ROPE will have broad impact, reaching instructors at
all types of colleges and universities who are teaching any standard
mathematics course in any of a number of different ways.  We further
expect that as ROPE develops we will be able to extend it to include other
disciplines, further increasing its impact.

\end{document}
