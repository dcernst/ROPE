\documentclass[11pt]{article}

\renewcommand{\rmdefault}{phv} % Arial
\renewcommand{\sfdefault}{phv} % Arial

\usepackage{graphicx}
\pagestyle{empty}

\oddsidemargin=0in
\textwidth=6.5in
\textheight=9in
\topmargin=-.625in
\linespread{1.4}

\begin{document}
\begin{center}
\Large\textbf{ Collaborative Research: Resource of Open\\
Problems for Education (ROPE) }
\end{center}

\begin{section}{Data Management Plan}

The types of data that are to be produced and managed in the course of
this project are those having to do with the problems that are being
stored in the ROPE database, and user data for registered users of the
site.  Data about problems will include problem and solution text and
metadata describing the problems.  User data will include usernames (as
the users provide them) and e-mail addresses, a system password,
information about problem ratings and comments they have entered into the
system and (optionally) the users' names and other contact information.
All of these data will be maintained in the ROPE database.

The security of the database will be provided for by the management of the
server.  Login access to the server will be limited to those requiring it
(system administrators and the programmers working on the project), and it
will be maintained with an up-to-date and secure Linux operating system.
Physical access to the server will be restricted by maintaining it in a
locked server room to which only authorized personnel will have access.

All problem and feedback data submitted by users will be governed by the
Terms of Service of ROPE, which will specify that ownership of the
submitted data is retained by the users and that all submitted material
will be managed under an appropriate open license, most likely a Creative
Commons attribution-share alike license.  (The choice of license will be
made by the PIs after consultation with the Advisory Group at the
beginning of the project.)  This will allow other users to take, modify
and use the problems in the library.

We do not anticipate the need to archive or remove data from the library's
database, and so expect that these data will continue to be accessible.

\end{section}

\end{document}

   1. the types of data, samples, physical collections, software, curriculum materials, and other materials to be produced in the course of the project;

   2. the standards to be used for data and metadata format and content (where existing standards are absent or deemed inadequate, this should be documented along with any proposed solutions or remedies);

   3. policies for access and sharing including provisions for appropriate protection of privacy, confidentiality, security, intellectual property, or other rights or requirements;

   4. policies and provisions for re-use, re-distribution, and the production of derivatives; and

   5. plans for archiving data, samples, and other research products, and for preservation of access to them.
