\documentclass[11pt]{article}
\pagestyle{empty}

\usepackage{graphicx}
\usepackage{color}
\usepackage[margin=1in]{geometry}

\linespread{1.2}

\begin{document}
\begin{center}
{\Large \textbf{Collaborative Research: Resource of Open\\
Problems for Education (ROPE)}}
\end{center}

\bigskip
\noindent
The proposed project will develop an online, electronic library, the
Resource of Open Problems for Education (ROPE), that will provide many
innovative, well-tested, and documented mathematics problems that
will be useful in a wide range of courses and assignment types.  Because
working problems is at the heart of student learning and engagement with
mathematics, this will be a tool that may promote student learning and be
of use to authors, instructors and students alike.  Any user will be able
to search ROPE for problems by keyword or by specific subject matter or
characteristics.  Problems will be contributed by the community of users
and will have associated descriptive information that will include user
feedback and comments, as well as statistics on the frequency of the
problem's use.  Users will be able to comment on and rate problems, and
flag them using a commonly understood ``endorse'' feature (similar to
those used on social networking sites).  They will also be able to create
and share collections of problems, and collections of these collections,
for later use.  These collections may be for specific assignments
(homework, quizzes, worksheets, etc.) and may be collected to describe
problems sets or assignments for entire courses.  ROPE will have passive
and active editorial management, the former by being able to sort search
results by ratings and usage and the latter by the community of users of
ROPE.

\bigskip\bigskip
\noindent
{\large \textbf{Intellectual Merit}}

The Resource of Open Problems for Education (ROPE) will address a current
need in undergraduate mathematics education: the need for a
widely-available source of high-quality, well documented problems that
instructors can use in a variety of educational venues.  Much of the
learning that takes place in mathematics is driven by students' work, and
the success of that learning is fundamentally dependent on the types of
problems on which they work.  ROPE will provide a resource that will be
\emph{free and open-source}, providing problems for users without
commercial entanglements.  Because it will have a powerful search
capability, users will be able to easily find problems that allow them to
address their needs for instruction or learning.  It will allow users to
create \emph{collections of problems}, and \emph{collections of
collections}.  This capability will allow many different use-cases: for
example, instructors may construct homework sets, quizzes, and groups of
these for entire courses; or may construct model courses with supporting
material that they can then share with colleagues; etc.  And it will
support a \emph{community of users} who may contribute problems, content
and feedback on problems, and who may share their work and problem
collections with others.  All of these characteristics, taken in sum, will
result in a widely accessible and useful resource that may have a
significant impact on mathematics education as a whole.

\bigskip\bigskip
\noindent
{\large \textbf{Broader Impacts}}

The Resource of Open Problems for Education (ROPE) will be an easy-to-use,
high-quality resource that provides something that teachers and learners
of mathematics need: a source of problems to evaluate and promote
learning.  Its broad impact will stem from its ability to be such a
resource for a wide range of users, from students looking for sample
problems to study from to mathematics instructors creating entire IBL
courses.  The PIs have extensive and varied contacts with faculty in a
large number of institutions and communities of faculty, including Project
NExT, those interested in inquiry-based learning, and open-content
authors, which will allow for effective dissemination to those groups most
likely to use the resource.  We therefore expect that ROPE will have broad
impact, reaching instructors at all types of colleges and universities who
are teaching any standard mathematics course in any of a number of
different ways.  We further expect that as ROPE develops we will be able
to extend it to include other disciplines, though that is not a goal of
this proposal.

\end{document}
