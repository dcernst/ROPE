% dbtag(author) = Dana Ernst
% dbtag(date) = 2013/07
% dbtag(difficulty) = easy
% dbtag(type) = symbolic
% dbtag(course) = problem solving
% dbtag(majortopic) = number theory
% dbtag(minortopic) = problem solving
% dbtag(relatedproblem) = {}
% dbtag(relatedproblem) = {}
% dbtag(relatedinsequence) = false
% dbtag(requiredtech) = none
% dbtag(keywords) = factors, perfect squares, fundamental theorem of arithmetic, unique factorization, factorization

Imagine a hallway with 1000 doors numbered consecutively 1 through 1000.  Suppose all of the doors are closed to start with.  Then some dude with nothing better to do walks down the hallway and opens all of the doors.  Because the dude is still bored, he decides to close every other door starting with door number 2.  Then he walks down the hall and changes (i.e., if open, he closes it; if closed, he opens it) every third door starting with door 3.  Then he walks down the hall and changes every fourth door starting with door 4.  He continues this way, making a total of 1000 passes down the hallway, so that on the 1000th pass, he changes door 1000.  At the end of this process, which doors are open and which doors are closed? 

\begin{hint}
Is door number 8 open or closed?  Why?  Is door number 9 open or closed?  Why?  How about 11 and 16?
\end{hint}

\begin{solution}
The open doors are the perfect squares less than 1000.  Why?  A door will end up in the open position if it gets changed an odd number of times.  Door $n$ gets changed on pass number $k$ exactly when $k$ is a factor of $n$.  The only doors with an odd number of factors are the perfect squares.
\end{solution}